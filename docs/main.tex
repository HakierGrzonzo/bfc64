% !TEX program = xelatex
%Wzór dokumentu
%tu zmień marginesy i rozmiar czcionki
\documentclass[a4paper,12pt]{article}
\usepackage{inputenc}[utf8]
\usepackage[margin=2.5cm]{geometry}

%Lepiej tego nie zmieniaj, jak co to dodawaj pakiety
\usepackage{titlesec}
\usepackage{titling}
\usepackage{fancyhdr}
\usepackage{mdframed}
\usepackage{graphicx}
\usepackage{amsmath}
\usepackage{amsfonts}
\usepackage{multicol}
\usepackage{listings}

%inny wygląd
%\usepackage{tgbonum}


%Zmienne, zmień je!
\graphicspath{ {./ilustracje/} }
\title{bfc64 - dokumentacja}
\author{Grzegorz Koperwas}
\date{\today}

%lokalizacja polska (odkomentuj jak piszesz po polsku)

\usepackage{polski}
\usepackage[polish]{babel} 
\usepackage{indentfirst}
\usepackage{icomma} 

\brokenpenalty=1000
\clubpenalty=1000
\widowpenalty=1000    

%nie odkometowuj wszystkiego, użyj mózgu
%\renewcommand\thechapter{\arabic{chapter}.}
\renewcommand\thesection{\arabic{section}.}
\renewcommand\thesubsection{\arabic{section}.\arabic{subsection}.}
\renewcommand\thesubsubsection{\arabic{subsubsection}.}

%Makra

\newcommand{\obrazek}[2]{
\begin{figure}[h]]
    \centering
    \includegraphics[scale=#1]{#2}
\end{figure}
}     
        

\newcommand{\twierdzonko}[1]{
    \begin{center}
    \begin{mdframed}
    #1
    \end{mdframed}          
    \end{center}
} 

\newcommand{\dwanajeden}[2]{
\ensuremath \left( \begin{array}{c}
    #1\\
    #2
\end{array} \right)
}  

%Stopka i head (sekcja której nie powinno się zmieniać)
\pagestyle{fancy}
\fancyhead{}
\fancyfoot{}

%Zmieniaj od tego miejsca
\rfoot{\thepage}
\lfoot{Grzegorz Koperwas}
\lhead{}
\rhead{Ostatnia edycja: \today}
\renewcommand{\headrulewidth}{1pt}
\renewcommand{\footrulewidth}{1pt}

\begin{document}
    \maketitle
    \thispagestyle{fancy}

    \section{Kompilowanie}

    \subsection*{Wymagania do kompilacji:}
    \begin{itemize}
        \item gcc - kompilator
        \item make
        \item python - generowanie pliku \texttt{template.cpp} (patrz kompilator)
        \item xelatex - dokumentacja
    \end{itemize}

    Kompilujemy poleceniem \texttt{make}.
    
    Instalujemy poleceniem \texttt{sudo make install}

    Dokumentacje kompilujemy poleceniem \texttt{make docs}.

    Przykładowy test uruchamiamy poleceniem \texttt{make test}, powinien uruchomić emulator Vice z programem.

    \subsubsection*{Wymagania do używania}

    \begin{itemize}
        \item kickassembler\footnote{http://theweb.dk/KickAssembler/Main.html\#frontpage} - bfc64 generuje pliki \texttt{.asm} dla tego assemblera
        \item System Linux, testowane wyłącznie na Arch'u
    \end{itemize}

    Program uruchamiamy poleceniem \texttt{bfc64 <ścieżka do pliku źródłowego>}, utworzy on plik a.asm gotowy do przetworzenia kickassemblerem za pomocą polecenia \texttt{kickass a.asm}. Utworzy on nam plik wykonywalny \texttt{a.prg} dla commodore 64 lub emulatora.

    Emulator Vice\footnote{https://vice-emu.sourceforge.io/} uruchomi automatycznie programy poleceniem:
    \begin{lstlisting}
    x64sc -autostart $PWD/a.prg
    \end{lstlisting}

    \section{Działanie kompilatora bfc64}

    Kompilator składa się z dwóch głównych części:
    \begin{itemize}
        \item Parsera - zamieniającego pliki tekstowe na listę symboli
        \item Kompilatora - zamieniającego listę symboli na kod assemblera korzystając z funkcji z przestrzeni \texttt{arch}
    \end{itemize}

    \subsection*{Parser}

    Parser jest funkcją \texttt{parseSourceFile}, która przyjmuje jako argument plik z kodem źródłowym. Zamienia ona wewnętrznie znaki języka \emph{brainfuck} na odpowiadające im symbole według tablicy \ref{tab:symbole}

    \begin{table}[h]
        \label{tab:symbole}
        \begin{tabular}{|c|l|}
            \hline
            Symbol         & Wartość   \\ \hline
            +              & inc       \\ \hline
            -              & dec       \\ \hline
            \textless{}    & left      \\ \hline
            \textgreater{} & right     \\ \hline
            {[}            & loopBegin \\ \hline
            {]}            & loopEnd   \\ \hline
            ,              & in        \\ \hline
            .              & out       \\ \hline
        \end{tabular}
        \centering
        \caption{Symbole oraz odpowiadające im wartości z \texttt{SymbolType}}
    \end{table}

    Dodatkowo parser wyświetla ostrzeżenia w przypadku jeśli w trakcie pętli za każdą iteracją jest porównywalna inna komórka pamięci, na przykład dla pętli \texttt{[>><]} zostanie wyświetlone ostrzeżenie wraz z numerem lini początkowym i końcowym.

    Jeśli parser napotka \texttt{]} bez poprzedniego \texttt{[} czy nie znajdzie w pliku końca pętli przed jego końcem zgłosi on błąd użytkownikowi i nie skompiluje programu.

    Parser traktuje wszystkie inne znaki jako komentarz.
    
    Dodatkowo parser może zostać rozbudowany o zoptymalizowane symbole \texttt{add, subtract, jmpLeft, jmpRight} które łączą sąsiednie symbole w jeden\footnote{Na przykład \texttt{++-+++} zostanie zamienione na \texttt{add}}, oszczędzając przy tym pamięć oraz cykle procesora, jednak w obecnej wersji nie zostało to jeszcze zaimplementowane. 

    \subsection*{Kompilator}

    Kompilator tworzy stringa na podstawie listy symboli z parsera za pomocą wzorców generowanych podczas kompilacji z plików w folderze \texttt{processor/arch/c64}. Dodatkowo na początek dołącza \texttt{arch::begin} a na koniec \texttt{arch::end}

    Wzorce, z których korzysta kompilator są generowane automatycznie skryptem pythona \texttt{templateGen.py}. Generuje on plik źródłowy \texttt{template.cpp} z pomocą pliku-wzorca \texttt{template.cpp.template}, do którego podstawia za placeholdery\footnote{W formie \texttt{\$nazwa}} odpowiednie sumy stringów oraz parametrów funkcji według plików-wzorców assemblera, w których podmienia \texttt{label()} na argument \texttt{label} itd. Opis wzorców w sekcji \ref{wzorce}.


    Wygenerowany \texttt{string} program zapisuje do pliku wyjściowego.

    \section{Opis działania skompilowanego programu}\label{wzorce}
    
    \subsubsection*{Wyzwania architektury 6502/6510/commodore 64}

    Procesory \emph{MOS 6502/6510} posiadają 8 bitową szynę danych oraz 16 bitową szynę adresową. Z tego powodu ,,pointery'' muszą się znajdować w pierwszych 256 bajtach pamięci, tak zwanej ,,zeropage''. Kompilator umieszcza w pamięci o adresie \texttt{\$00FB} adres \texttt{\$C000}, który jest adresem początku taśmy. 

    Procesor posiada opcje indeksowania pamięci rejestrem X lub Y, taki odpowiednik $\left(\texttt{pointer} + \texttt{rejestr}_X \right)^*$ w języku \texttt{C++}. Jednak jako iż rejestr \texttt{X} jest 8 bitowy, a chcemy taśmę o długości większej od 256 to ruch po taśmie w lewo wygląda tak:

    \begin{figure}[h]
        $\vdots$
    \centering
        \begin{lstlisting}
    cpx #$00 // compare X and 0
    bne label // if X != 0, go to label
    dec $00fb + 1 // decrement most significant 
                  // byte of tape pointer by 1
label:
    dex // decrement X by 1
        \end{lstlisting}
        $\vdots$
    \centering
    \end{figure}

    \subsection*{Taśma}

    Definicja taśmy, w pliku \texttt{end.asm}, wygląda tak:

    \begin{figure}[h]
        \begin{lstlisting}
*=$C000 "Tape" // at address $C000

.fill 1024, 0 // place 1024 zeros
        \end{lstlisting}
    \end{figure}

    Generuje ona taśmę o długości 1024 bajtów, jednak jako iż kompilator nie zabezpiecza nas przed ,,wyjściem'' z jej przestrzeni, zatem jeśli komuś chce się pisać dużo \texttt{>} i wiedząc iż adres pierwszej komórki to \texttt{\$C0FF} możemy zmieniać kolory tła, tekstu, odtwarzać muzykę oraz nadpisywać program.

    \subsection*{Wypisywanie znaków na ekran i ich odczyt}

    Kernal commodore 64 posiada \emph{funkcje}, które realizują te zadania.

    Pod adresem \texttt{\$FFCF} jest funkcja, która umieszcza w rejestrze A wartość wprowadzoną przez użytkownika.

    Pod adresem \texttt{\$FFD2} jest funkcja, która wartość w rejestrze A wypisuje na ekran.

    Wadą tych funkcji jest to iż nie używają zestawu znaków ASCII, tylko własnego PETSCII\footnote{https://www.c64-wiki.com/wiki/PETSCII}. Z tego powodu przy wypisywaniu na ekran za pomocą \texttt{.} trzeba ustawiać wartość komórki na wartości PETSCIII
    \end{document}